\documentclass[helvetica,english,logo,notitle,totpages,utf8]{europecv2013}
\usepackage{graphicx}
\usepackage[a4paper,top=1.2cm,left=1.2cm,right=1.2cm,bottom=2.5cm]{geometry}
\usepackage[english]{babel}
\usepackage[T1]{fontenc}
\usepackage{natbib}
\usepackage{bibentry}


\ecvname{Menelaos Perdikeas}
\ecvaddress{7 Milton St, Arlington MA 02474, United States}
%\ecvtelephone[+44 2012345679]{+44 7123456789}
\ecvemail{mperdikeas@gmail.com}
\ecvgithub{mperdikeas}
%\ecvstackoverflow{Marcus Junius Brutus}
\ecvhomepage{http://mperdikeas.github.io/}
\ecvgender{Male}
\ecvdateofbirth{24 December 1974}

%\ecvfootnote{© European Union, 2002-2013 | http://europass.cedefop.europa.eu}

%\ecvbeforepicture{\raggedleft}
%\ecvpicture[width=2.5cm]{fototessera}
%\ecvafterpicture{\ecvspace{-37mm}}

% my own additions to the template - start
\newcommand{\technologies}[0]{\textbf{\textit{Technologies:}}}
\usepackage{multicol} % Required for multiple columns of text
% my own additions to the template - end

\begin{document}
\selectlanguage{english}

\begin{europecv}
\ecvpersonalinfo[10pt]


\ecvitem[10pt]{SUMMARY OF CAREER}{
  I am a full stack software engineer able to design and code all layers of a modern application: browser, application server, database.
  I have worked with a variety of technologies in all tiers. In the browser: from vanilla HTML/CSS/JavaScript, jQuery to using UI component
  libraries like jQuery UI or DHTMLX, to Webpack / ReactJS. In the application server I've experience with a Java stack
  deployed in Tomcat or JBoss servers with both presentation oriented (Spring Web MVC / JSP) and
  service-oriented (JAX-RS, SOAP, Restlet, Jersey, RESTEasy) applications. In the database I have experience with Sybase, Oracle,
  MySQL and PostgreSQL databases.
  I am a proficient Emacs user (with an Emacs setup that I've honed over the years for maximum productivity)
  and I like to live as close to the command line as possible
  (though I can of course also use an IDE
  if required). In addition to writing the code I can
  handle the build system (Ant / Ivy), unit testing, static code analysis, coding standards and automatic compliance checking
  , code coverage,
  integration testing, shell-scripting, etc.
  I am also proficient  in JavaScript-the-language (ECMAScript 6) including
  the Node / Babel / Webpack ecosystem and the Mocha (unit tests) and FlowType (static code analysis) tools.


  My career spans almost 2 decades of softwrare development and technical project management in all manners of companies
  from start-ups and SMEs to world class organizations
  such as the \href{http://www.esa.int/About_Us/ESAC}{European Space Agency} and the \href{https://www.cfa.harvard.edu/}{Harvard-Smithsonian Center for Astrophysics} (my current position).

  In my free time I study Math and develop JavaScript games which you can play at:
  \href{https://mperdikeas.github.com/games.html}{https://mperdikeas.github.com/games.html}.  

  }


\ecvposition{Job applied for}{Information and communication technology, FG IV}

\ecvsection{Work experience}

\ecvworkexperience{October 2016 -- Present}{SW Application Specialist}{\href{https://www.cfa.harvard.edu/}{Harvard-Smithsonian Center for Astrophysics}}{60 Garden St, Cambridge, MA 02138 (USA)}
                  {I am a member of the Data Archive team in the High Energy Astrophysics Division of the CfA (Center for Astrophysics),
    at the Smithosonian Astrophysical Observatory (Cambridge MA). I am coding primarily in Java and JavaScript but also in Python.
    Projects and accomplishments in this position:
    \begin{itemize}
      \item Designed and implemented a test suite for both the front-end (presentation-oriented) and the back-end (service-oriented)
    components of the new Chandra Proposer system (this is the system to be used by researchers wishing to submit proposals for
    use rights on the Chandra X-Ray telescope). The test suite allows us to simulate a large number of concurrent users
    interacting with the web site and is used to stress-test both the web application itself and also the login functionality
    offered by the SSO mechanism
    (see below)
    \technologies Python 2.7 (with the Requests library), Java
  \item Designed and implemented the signup and edit profile pages for the new Chandra Proposer system.
      \technologies HTML/CSS/DHTMLX/JavaScript, Java, Spring Web MVC, JSP, JAX-RS (RESTEasy), Sybase.    
  \item Designed and implemented an SSO (Single-Sign On) solution loosely based on the
    \href{https://en.wikipedia.org/wiki/Central_Authentication_Service}{CAS protocol}. The system provides
    an \href{https://en.wikipedia.org/wiki/Role-based_access_control}{RBAC}-based Authorization
    and Authentication in a uniform manner for all participating applications. It also supports all features
    taken for granted in a modern authentication system (salted and peppered cryptographic hashes, auditing, etc.).
      \technologies HTML/CSS/DHTMLX/JavaScript, Java, JSP, JAX-RS (RESTEasy), Sybase.
    \item Proactively advocated and implemented a new team-wide build system for Java WAR
      and JAR applications and libraries . The new build system is based on Ant and Ivy
      (with a local file-system based repository) and guarantees
    fully reproducible builds with no Internet connectivity requirements (i.e. no dependencies are accessed or downloaded during
    the build). The new build system
    additionally includes the following features: (a) integrated testing (via JUnit) (b) detection of bugs by means
    of static code analysis using two different but complementary solutions: FindBugs and PMD (c) style checker using
    CheckStyles to ensure code styling conformance. 
      \technologies{} Ant, Ivy, FindBugs, PMD, CheckStyles
    \end{itemize}
                  }

 \ecvworkexperience{February 2016 -- September 2016}{Lead Programmer and Technical Project Manager}
                   {\href{http://www.esac.com}{European Space Agency - ESAC} (through \href{http://www.neuropublic.gr}{Neuropublic})}
                   {Methonis 6, Pireaus 18545, Greece}
                   {
                     I was the lead coder and technical project manager for the RAWDAR/RADACER project which aimed
    to create a data clearing house for the telemetry of a large number (upwards of 12) of European Space Agency
    missions. The work was for the {\href{http://www.esac.com}{ESAC}} establishment of ESA though, unlike the previous project, it was
    mostly off-site (which is why I indicate the address of Neuropublic in the address field above). In this project:
  \begin{itemize}
    \item I studied telemetry and telecontrol standards for recent ESA missions.
    \item coded a library to decode BSCS telemetry files for the BepiColombo mission telemetry format. This was done mostly to gain
      familiarity with the subject domain and the relevant standards.
    \item designed and implemented a prototype REST interface to be exposed by the unified data archive.
    \item designed and implemented the database and the REST server back-end of a graphical web application that displays statistics
      on the accumulated telemetry for the various missions.
  \end{itemize}
  \technologies Java, ReactJS, JAX-RS, PostgreSQL 9.2, PL/pgSQL, JDBC, SQL.
                   }
                   
                   \ecvworkexperience{March 2013 -- January 2016}{Software Architect and Lead Developer}{\href{http://www.esac.com}{European Space Agency -- ESAC} (through \href{http://www.neuropublic.gr}{Neuropublic})}
                                     {Villanueva de la Cañada
                                     E-28692 Madrid, Spain}
                                     {
I was the software architect and lead coder for the
    \href{registry.euro-vo.org}{new EuroVO Registry system}
    which is scheduled to replace the \href{registry.euro-vo.org}{existing EuroVO Registry system} that had been
    serving the global \href{http://www.ivoa.net/}{IVOA} community for the last ten (10) years.

  I worked for three years with people from the \href{http://www.sciops.esa.int/index.php?project=SAT}{Science
    Archives Team at ESAC}, with little or no supervision
  and designed and implemented the entire system from scratch with the exception of most
  of the front-end web-based GUI (done in \href{http://www.gwtproject.org/}{GWT}).
  Among others, I designed and implemented:
  \begin{itemize}
  \item the \href{http://www.openarchives.org/pmh/}{OAI-PMH} harvesting functionality. I couldn't find a reliable 3rd party OAI-PMH library
    so I implemented the entire protocol from scratch.
    \technologies{} Java, XPath 2.0 (Saxon), XSLT, REST (Jersey), HTTP, Tomcat 7
  \item the \href{http://www.ivoa.net/documents/RegTAP/}{RegTAP} functionality and interface. I came into the project with \textbf{zero} domain-specific
    knowledge and I managed to implement the second, globally, RegTAP search interface and make material
    contributions to the RegTAP specification itself to the effect that I was recognized as a co-author of the spec.
    \technologies{} Java, XPath 2.0 (Saxon), REST (Jersey), JDBC, Tomcat 7
  \item the \href{http://www.ivoa.net/documents/RegistryInterface/20091104/REC-RegistryInterface-1.0.pdf}{IVOA RI1} Search interface. This is a \href{https://en.wikipedia.org/wiki/SOAP}{SOAP interface}, defined in WSDL (i.e. following a top-down approach). 
    Due to the complexity of the XSDs used by the IVOA community I ran into problems with the automatic stub
    generation by tools like \href{https://docs.oracle.com/javase/6/docs/technotes/tools/share/xjc.html}{XJC} (which I have documented in \href{http://stackoverflow.com/q/17265960/274677}{SO} and raised
    against the tool's \href{https://java.net/jira/browse/JAXB-965}{JIRA}). So I decided to implement
    my own \href{https://docs.oracle.com/javase/6/docs/technotes/tools/share/wsimport.html}{wsimport}-like tool that reads WSDL and generates Java code (stubs and skeletons) to perform SOAP calls.
    \technologies{} Java, SOAP (Jersey), XSD, template-based code generation with \href{http://www.stringtemplate.org/.}{String Template}
  \item the \href{https://en.wikipedia.org/wiki/XML_Schema_(W3C)}{XML XSD schema-validation}
    and the IVOA validation of the various services.
    \technologies{} Java, XML, XSD, XPath 2.0 (Saxon), HTTP.
  \item the report, email notifications and visual plots generation subsystem.
    \technologies{} Java, \href{http://www.jfree.org/jfreechart/}{JFreeChart}
    \item the entire database schema for PostgreSQL. I wrote the DDL statements for the entire database which comprises 102 tables in 4 schemas and was responsible for the installation, fine-tuning, index creation, performance optimization, etc of the PostgreSQL database. Wrote the entire DB-facing \href{https://en.wikipedia.org/wiki/Database_abstraction_layer}{DAO} code in Java.
      \technologies{} PostgreSQL 9.2, PL/pgSQL, JDBC, SQL.
    \item Finally, I also handled all DevOps aspects including: installing and configuring Apache Tomcat7 and the PostgreSQL cluster, authoring the entire build system based on
      Ant and \href{http://ant.apache.org/ivy/}{Ivy}, writing test cases (JUnit), coding standards checking with \href{http://checkstyle.sourceforge.net/}{Checkstyle}, source code analyzers with
      \href{https://pmd.github.io/}{PMD} and \href{http://findbugs.sourceforge.net/}{FindBugs},
      code coverage with \href{http://cobertura.github.io/cobertura/}{Cobertura}, etc.
      \end{itemize}
                                     }



 \ecvworkexperience{June 2012 -- March 2013}{Senior Software Engineer}
                   {\href{http://www.neuropublic.com}{Neuropublic S.A.}}
                   {Methonis 6, Pireaus 18545, Greece}
                   {
    I worked as a software engineer in the company's flagship \href{https://www.neuropublic.gr/en/component/k2/gaia-epicheirein}{GAIA} series of cloud-based services.
    Among the things I did:
    \begin{itemize}
    \item designed and developed all the server-side code for an expert system that advises farmers
      on the subsidies they are entitled to apply for. I devised a formal way to capture the legal requirements
      for each subsidy in the form of \href{https://en.wikipedia.org/wiki/S-expression}{Lisp S-expressions}.
      Essentially, each subsidy ministerial decision is
      modelled as a declarative, Lisp-like \href{https://en.wikipedia.org/wiki/Domain-specific_language}{DSL}.
      Then the system reads from the database the profile of each user (age, income, past subsidies, location
      and size of fields, crops, animal capital) and dynamically evaluates them against all possible subsidies.
      The system finally informs the user which subsidies (s)he can apply for and also creates
      a visual diagram that explains to the users why they failed to qualify for certain subsidies.
      S-expressions were used in order to allow for dynamic evaluation and also because the company wanted
      the DSL logic / script to be created by business people. Therefore, Lisp S-expressions were chosen
      for their reduced mental baggage, as Lisp requires one to use no syntax except for opening and closing parenthesis.
      \technologies{} \href{https://clojure.org/}{Clojure} (for
      the dynamic evaluation of the S-expressions), dynamic evaluation of Clojure code from Java,
      \href{https://en.wikipedia.org/wiki/DOT_(graph_description_language)}{DOT} (for visualizing
      criteria logic using graphs), \href{https://en.wikipedia.org/wiki/IText}{iText} for creating
      PDF reports, \href{http://www.stringtemplate.org/}{StringTemplate}, JDBC.
    \item designed and developed the authentication / authorization database and implemented all back-end Java
      code to integrate Apache Shiro into our \href{http://www.javaserverfaces.org/}{JSF} pages and deliver an elaborate RBAC authentication solution
      for all GAIA applications.
      \technologies{} \href{http://shiro.apache.org/}{Apache Shiro}, PostgreSQL, JDBC.
    \item implemented a \href{http://www.jython.org/}{Jython} solution integrating Java code with
      Python scripts. The objective was to
      allow the accounting people to write simple arithmetic expressions (formulas) to derive dynamically
      various fields that were used in the generation of financial reports.
      \technologies{} \href{http://www.jython.org/}{Jython}, Python, \href{http://community.jaspersoft.com/project/jasperreports-library}{Jasper Reports}.
    \end{itemize}
                   }


 \ecvworkexperience{February 2010 -- March 2012}{Senior Software Engineer}
                   {\href{http://www.synelixis.com}{Synelixis Solutions}}
                   {157 Perissou \& Chalkidos, 14343, Athens, Greece}
                   {
                     I led the company's technical participation in the \href{http://www.synelixis.com/coast/}{COAST} project.
                     The domain of the project was multimedia
    content delivery and streaming over a dynamically configured mesh of network caches that are co-located
    with network routers (something \textit{like} a content delivery network).
    Some of the things I did:
    \begin{itemize}
    \item designed and implemented: (a) the cache communication protocol on top of TCP and (b) custom HTTP proxies
      in Java that transparently take advantage of the caches. Also, for demonstration purposes, coded
      a few pages for video streaming solutions.
       \technologies{} Java TCP networking, HTML, integration with video streaming / playback.
    \item designed and implemented a rich client overview and monitoring application that visualizes
      the utilization and available capacity of all network caches, plots graphs over time and allows
      an operator to dynamically interact with them by issuing commands or writing Python scripts
      (to automate common administrative logic) in a console-like prompt.
       \technologies Java (networking), Java Swing, Jython (for the dynamic console)
    \end{itemize}
                   }

 \ecvworkexperience{January 2009 -- October 2010}{Senior Software Engineer}
                   {\href{http://www.synelixis.com}{Synelixis Solutions}}
                   {10 Farmakidou, 34100, Chalkida, Greece}
                   {
    I led the company's technical participation in the \href{http://www.innovationseeds.eu/Virtual_Library/Results/BEYWATCH.kl}{BeyWatch} project. The domain of the project was home automation
    and centralized optimization and scheduling of household tasks involving use of electrical appliances
    in a system of continuously changing real-time tariffs.
    Some of the things I did:
    \begin{itemize}
    \item implemented a brute-force particle swarm optimization algorithm in Java that accepts as inputs
      a set of tasks, electricity tariffs over the next 24 hours and a weather forecast (so that solar
      panel output or heating needs may be anticipated) and calculates the best possible scheduling of
      tasks to minimize either cost or carbon footprint or a combination of both.
      \technologies{} Java
    \item implemented a rich client Java application that visualizes the scheduling of the various
      tasks against the electricity tariffs
      \technologies{} Java Swing
    \item implemented the client (scheduler) side of a number of controllers that are used to program or monitor
      household appliances (washing machines, refrigerators, solar panels)
      \technologies{} Java, REST, \href{https://www.osgi.org/}{OSGI}.
    \end{itemize}
                   }


 \ecvworkexperience{March 2008 -- September 2008}{Technical Project Manager}
                   {\href{http://www.semantix.gr}{Semantix Information Technologies SA.}}
                   {62 Konstantinou Tsaldari, 11476, Athens, Greece}
                   {
                     I was technical project manager for the EGOS Visualization Tool which was a contract for the ESOC
                     establishment of the European Space Agency. The project implemented a graphical front-end
                     that integrates and provides additional functionality on top of the existing tools used by ESOC to check
                     compliance with coding standards and conventions.
                   }

 \ecvworkexperience{2005 -- 2007}{Software Engineer}
                   {\href{http://www.semantix.gr}{Semantix Information Technologies SA.}}
                   {62 Konstantinou Tsaldari, 11476, Athens, Greece}
                   {
    I was responsible for the implementation of the critical 
    conversions functionality in both the \href{http://www.tapeditor.com/}{Roaming Studio} and the Roaming Components products. These were
    the company's flagship telecom products and revolve around the processing of TAP
    files which are used in \href{https://en.wikipedia.org/wiki/Roaming}{GSM Roaming}.
    The conversion logic is responsible for converting, e.g., a TAP file of version TAP3.11 into a file
    of version TAP3.10. The files are defined using
    different \href{https://en.wikipedia.org/wiki/Abstract_Syntax_Notation_One}{ASN.1} grammars both from
    a syntax and a semantics point of view so
    the transformation is a complex business logic procedure. Since there are more than 7 different \href{http://what-when-how.com/roaming-in-wireless-networks/transferred-account-procedures-billing-and-settlement/}{TAP3}
    versions and conversion had to take place between any arbitrary pair there was
    a large number of conversion routines to be written. I defined a
    \href{https://en.wikipedia.org/wiki/Domain-specific_language}{DSL} used to describe the logic
    behind these conversions (essentially transformation algorithms on deeply nested ASN.1 trees) and built a Python
    script that automatically generated C++ code implementing these transformations. In total I was able to
    reduce tens of thousands of lines of C++ code into as little as 700 conversion rules expressed in the
    above DSL
    (plus the Python script code generator).
    \technologies{} ASN.1, C++, Python
                   }                   

 \ecvworkexperience{2005 -- 2007}{Software Engineer}
                   {\href{http://www.semantix.gr}{Semantix Information Technologies SA.}}
                   {62 Konstantinou Tsaldari, 11476, Athens, Greece}
                   {
    I was part of the engineering team that was tasked to implement the new Vodafone Greece billing system (project ATLAS).
    Data (customer, subscriber, bills and
    calls information) migration to the new billing system and integration with over a dozen legacy
    peripheral systems which were scheduled to survive the migration to the new billing system.
    I was responsible for the integration code which was implemented in
    C++ and Java (for higher-level daemons) and which undertook to sustain the information flow between the
    new billing system and legacy systems which were not going to be replaced. Since no modifications were
    permitted in the legacy systems, the new billing system had to be wrapped in a fa\c{c}ade allowing it to
    expose the same interface towards the legacy systems.
    I was also involved with the design of a massive ``mediating'' staging database of more
    than 400 tables and 40,000 lines of \href{https://en.wikipedia.org/wiki/PL/SQL}{PL-SQL} code
    (a large percentage of which I implemented myself).
    Finally I had small exposure with \href{https://en.wikipedia.org/wiki/Portal_Software}{Portal Infranet} programming
    (now \href{http://www.orafaq.com/wiki/BRM}{Oracle Billing and Revenue Management}).
    \technologies{} Java, Oracle, PL-SQL, C++, C
                   }                   

 \ecvworkexperience{2001 -- 2002}{Software Engineer}
                   {\href{http://www.semantix.gr}{Semantix Information Technologies SA.}}
                   {62 Konstantinou Tsaldari, 11476, Athens, Greece}
                   {
    Design and implementation of the Vodafone Greece Data Clearing House system for roaming calls.
    I implemented code in C++ (for the decoding / encoding and validations / transformations
    of the \href{http://what-when-how.com/roaming-in-wireless-networks/transferred-account-procedures-billing-and-settlement/}{TAP} records), Java (for the orchestration logic) and PL/SQL (for server-side processing)
    \technologies{} C++, ASN.1, Java, Oracle, PL-SQL.
                   }                   
                   
%
%
%
% TODO: I am left to continue with the education and training section
%
%
%
%

\ecvsection{Education and training}

\ecveducation{1997 -- 2001}{PhD - Thesis Title: `Young People in the Construction of the Virtual University', Empirical research on e-learning}{Brunel University, London United Kingdom}{}{}

\ecveducation{1993 -- 1997}{Bachelor of Science in Sociology and Psychology}{Brunel University, London United Kingdom}{
- sociology of risk\par
- sociology of scientific knowledge / information society\par
- anthropology\par
E-learning and Psychology\par
- research methods}{}

\ecvsection{Personal skills}

\ecvmothertongue[20pt]{English}
\ecvlanguageheader
\ecvlanguage{French}{C1}{C2}{B2}{C1}{C2}
\ecvlastlanguage{German}{A2}{A2}{A2}{A2}{A2}
\ecvlanguagefooter[10pt]

\ecvitem[10pt]{Communication skills}{
- team work: I have worked in various types of teams from research teams to national league hockey. For 2 years I coached my university hockey team\par
- mediating skills: I work on the borders between young people, youth trainers, youth policy and researchers, for example running a 3 day workshop at CoE Symposium `Youth Actor of Social Change', and my continued work on youth training programmes\par
- intercultural skills: I am experienced at working in a European dimension such as being a rapporteur at the CoE Budapest `youth against violence seminar' and working with refugees.}

\ecvitem[10pt]{Organisational / managerial skills}{
- whilst working for a Brussels based refugee NGO `Convivial' I organized a `Civil Dialogue between refugees and civil servants at the European Commission 20th June 2002\par
- during my PhD I organised a seminar series on research methods}

\ecvitem[10pt]{Computer skills}{Creating pieces of Art and visiting Modern Art galleries. Enjoy all sports particularly hockey, football and running. Love to travel and experience different cultures.}

\ecvitem[10pt]{Other skills}{Creating pieces of Art and visiting Modern Art galleries. Enjoy all sports particularly hockey, football and running. Love to travel and experience different cultures.}

\ecvitem[10pt]{Driving licence}{A, B}


\bibliographystyle{plain}
\nobibliography{publications-en}

\ecvsection{ADDITIONAL INFORMATION}

\ecvitem{Publications}{\bibentry{pub1}}

\end{europecv}
\end{document} 
