%%%%%%%%%%%%%%%%%%%%%%%%%%%%%%%%%%%%%%%%%
% Cies Resume/CV
% LaTeX Template
% Version 1.1 (20/7/14)
%
% This template has been downloaded from:
% http://www.LaTeXTemplates.com
%
% Original author:
% Cies Breijs (cies@kde.nl)
% https://github.com/cies/resume with extensive modifications by:
% Vel (vel@latextemplates.com)
%
% License:
% CC BY-NC-SA 3.0 (http://creativecommons.org/licenses/by-nc-sa/3.0/)
%
%%%%%%%%%%%%%%%%%%%%%%%%%%%%%%%%%%%%%%%%%

%----------------------------------------------------------------------------------------
%	PACKAGES AND OTHER DOCUMENT CONFIGURATIONS
%----------------------------------------------------------------------------------------

\documentclass[10pt,a4paper]{article} % Font size (10-12pt) and paper size (a4paper, letterpaper, legalpaper, etc)

% Copyright (c) 2012 Cies Breijs
%
% The MIT License
%
% Permission is hereby granted, free of charge, to any person obtaining a copy
% of this software and associated documentation files (the "Software"), to deal
% in the Software without restriction, including without limitation the rights
% to use, copy, modify, merge, publish, distribute, sublicense, and/or sell
% copies of the Software, and to permit persons to whom the Software is
% furnished to do so, subject to the following conditions:
%
% The above copyright notice and this permission notice shall be included in
% all copies or substantial portions of the Software.
%
% THE SOFTWARE IS PROVIDED "AS IS", WITHOUT WARRANTY OF ANY KIND, EXPRESS OR
% IMPLIED, INCLUDING BUT NOT LIMITED TO THE WARRANTIES OF MERCHANTABILITY,
% FITNESS FOR A PARTICULAR PURPOSE AND NONINFRINGEMENT. IN NO EVENT SHALL THE
% AUTHORS OR COPYRIGHT HOLDERS BE LIABLE FOR ANY CLAIM, DAMAGES OR OTHER
% LIABILITY, WHETHER IN AN ACTION OF CONTRACT, TORT OR OTHERWISE, ARISING FROM,
% OUT OF OR IN CONNECTION WITH THE SOFTWARE OR THE USE OR OTHER DEALINGS IN THE
% SOFTWARE.

%%% LOAD AND SETUP PACKAGES

\usepackage[margin=0.75in]{geometry} % Adjusts the margins

\usepackage{multicol} % Required for multiple columns of text

\usepackage{mdwlist} % Required to fine tune lists with a inline headings and indented content

\usepackage{relsize} % Required for the \textscale command for custom small caps text

\usepackage{hyperref} % Required for customizing links
\usepackage{xcolor} % Required for specifying custom colors
\definecolor{dark-blue}{rgb}{0.15,0.15,0.4} % Defines the dark blue color used for links
\hypersetup{colorlinks,linkcolor={dark-blue},citecolor={dark-blue},urlcolor={dark-blue}} % Assigns the dark blue color to all links in the template

\usepackage{tgpagella} % Use the TeX Gyre Pagella font throughout the document
\usepackage[T1]{fontenc}
\usepackage{microtype} % Slightly tweaks character and word spacings for better typography

\pagestyle{empty} % Stop page numbering

%----------------------------------------------------------------------------------------
%	DEFINE STRUCTURAL COMMANDS
%----------------------------------------------------------------------------------------

\newenvironment{indentsection} % Defines the indentsection environment which indents text in sections titles
{\begin{list}{}{\setlength{\leftmargin}{\newparindent}\setlength{\parsep}{0pt}\setlength{\parskip}{0pt}\setlength{\itemsep}{0pt}\setlength{\topsep}{0pt}}}{\end{list}}

\newcommand*\maintitle[2]{\noindent{\LARGE \textbf{#1}}\ \ \ \emph{#2}\vspace{0.3em}} % Main title (name) with date of birth or subtitle

\newcommand*\roottitle[1]{\subsection*{#1}\vspace{-0.3em}\nopagebreak[4]} % Top level sections in the template

\newcommand{\headedsection}[3]{\nopagebreak[4]\begin{indentsection}\item[]\textscale{1.1}{#1}\hfill#2#3\end{indentsection}\nopagebreak[4]} % Section title used for a new employer

\newcommand{\headedsubsection}[3]{\nopagebreak[4]\begin{indentsection}\item[]\textbf{#1}\hfill\emph{#2}#3\end{indentsection}\nopagebreak[4]} % Section title used for a new position

\newcommand{\bodytext}[1]{\nopagebreak[4]\begin{indentsection}\item[]#1\end{indentsection}\pagebreak[2]} % Body text (indented)

\newcommand{\inlineheadsection}[2]{\begin{basedescript}{\setlength{\leftmargin}{\doubleparindent}}\item[\hspace{\newparindent}\textbf{#1}]#2\end{basedescript}\vspace{-1.7em}} % Section title where body text starts immediately after the title

\newcommand*\acr[1]{\textscale{.85}{#1}} % Custom acronyms command

\newcommand*\bull{\ \ \raisebox{-0.365em}[-1em][-1em]{\textscale{4}{$\cdot$}} \ } % Custom bullet point for separating content

\newlength{\newparindent} % It seems not to work when simply using \parindent...
\addtolength{\newparindent}{\parindent}

\newlength{\doubleparindent} % A double \parindent...
\addtolength{\doubleparindent}{\parindent}

\newcommand{\breakvspace}[1]{\pagebreak[2]\vspace{#1}\pagebreak[2]} % A custom vspace command with custom before and after spacing lengths
\newcommand{\nobreakvspace}[1]{\nopagebreak[4]\vspace{#1}\nopagebreak[4]} % A custom vspace command with custom before and after spacing lengths that do not break the page

\newcommand{\spacedhrule}[2]{\breakvspace{#1}\hrule\nobreakvspace{#2}} % Defines a horizontal line with some vertical space before and after it % Include structure.tex which contains packages and document layout definitions

\hyphenation{Some-long-word} % Specify custom hyphenation points in words with dashes where you would like hyphenation to occur, or alternatively, don't put any dashes in a word to stop hyphenation altogether

\newcommand{\technologies}[0]{\textbf{\textit{Technologies:}}}
\newcommand{\ts}{\textsuperscript}
\usepackage[utf8x]{inputenc}
\usepackage[greek, english]{babel}

\usepackage{fontspec}
\setmainfont{GFS Artemisia}
\begin{document} 

\newcommand{\gr}{\selectlanguage{greek}}
\newcommand{\en}{\selectlanguage{english}}
%----------------------------------------------------------------------------------------
%	NAME AND CONTACT INFORMATION
%----------------------------------------------------------------------------------------

\maintitle{Μενέλαος Περδικέας}{%December 24, 1974
}  % Your name and date of birth or subtitle

\noindent\href{mailto:mperdikeas@gmail.com}{mperdikeas@gmail.com}\bull % Your email address
%\textsmaller{+}30 (210) 5322914\bull
menelaos.perdikeas \textit{(Skype)}\bull % Your phone number(s) and Skype username
\href{http://mperdikeas.github.com}{mperdikeas.github.com}\\ % Your URL
Δήμητρας 11 \bull Αγ. Παρασκευή, 15342\bull Αθήνα\bull Ελλάδα % Your address

\spacedhrule{0.9em}{-0.4em} % Horizontal rule - the first bracket is whitespace before and the second is after

%----------------------------------------------------------------------------------------
%	SUMMARY SECTION
%----------------------------------------------------------------------------------------

\roottitle{Σύνοψη} % Root section title

\vspace{-1.3em} % Reduce whitespace after the Summary heading and the two-column content

\begin{multicols}{2}  % Start a two-column layout
  \noindent
  %\textit{A summary of your interests, achievements, history, topic of study or any other short summary of your professional life.}\\\\
  Προγραμματίζω σε Java από την έκδοση 1.1 που κυκλοφόρησε το 1997. Στην διάρκεια του χρόνου έχω συγγράψει
  πάνω από 300 χιλιάδες γραμμές κώδικα σε Java και έχω εξερευνήσει ή χρησιμοποιήσει έναν μεγάλο αριθμό
  από τεχνολογίες και εργαλεία στο συγκεκριμένο οικοσύστημα.
  Είμαι προχωρημένος χρήστης του Emacs με ένα setup που έχω τελειοποιήσει όλα αυτά τα χρόνια
  για την μέγιστη δυνατή παραγωγικότητα. Ζω στην γραμμή εντολών αλλά μπορώ να χρησιμοποιήσω και ένα
  IDE εάν απαιτηθεί.

  Έχω εργαστεί επαγγελματικά πρακτικά σε όλο το stack: service-oriented web services, presentation-oriented
  web applications, rich client applications
  server-side κώδικας Java, back-end σχεσιακές βάσεις δεδομένων.
  I am also typically involved in devops stuff as well. That is, in addition to writing the code I can
  handle the build system (Ant / Ivy), test cases, coding standards, compliance checking, code coverage,
  integration testing, shell-scripting, etc.
  HTML in the browser is the only part of the stack where I haven't worked in a professional capacity
  though I am very proficient in JavaScript-the-language (ECMAScript 6) including
  the Node and Babel ecosystem and have
  experience in HTML/CSS3/JavaScript in the browser (using WebPack etc.) 
  (check my \href{http://mperdikeas.github.com}{my github page} for some sandbox projects) --- I am also currently exploring ReactJS in my free time.

\end{multicols}

\spacedhrule{0.5em}{-0.4em} % Horizontal rule - the first bracket is whitespace before and the second is after

%----------------------------------------------------------------------------------------
%	EXPERIENCE SECTION
%----------------------------------------------------------------------------------------

\roottitle{Experience} % Top level section

\headedsection % Employer name which can include a hyperlink and location/URL on the right side of the page
    {\href{http://www.esac.com}{ESAC} (through \href{http://www.neuropublic.gr}{Neuropublic})}
{\textsc{European Space Agency / ESAC, Madrid, Spain}} {

\headedsubsection % Job title entry for the current employer
{Software Architect and Lead Programmer}
{March '13 -- January '16}
{\bodytext{I was the software architect and lead coder for the
    \href{registry.euro-vo.org}{new EuroVO Registry system}
    which is scheduled to replace the \href{registry.euro-vo.org}{existing EuroVO Registry system} that had been
    serving the global \href{http://www.ivoa.net/}{IVOA} community for the last ten (10) years.

  I worked for three years with people from the \href{http://www.sciops.esa.int/index.php?project=SAT}{Science
    Archives Team at ESAC}, with little or no supervision
  and designed and implemented the entire system from scratch with the exception of most
  of the front-end web-based GUI (done in \href{http://www.gwtproject.org/}{GWT}).
  Among others, I designed and implemented:
  \begin{itemize}
  \item the \href{http://www.openarchives.org/pmh/}{OAI-PMH} harvesting functionality. I couldn't find a reliable 3rd party OAI-PMH library
    so I implemented the entire protocol from scratch.\\
    \technologies{} Java, XPath 2.0 (Saxon), XSLT, REST (Jersey), HTTP, Tomcat 7
  \item the \href{http://www.ivoa.net/documents/RegTAP/}{RegTAP} functionality and interface. I came into the project with \textbf{zero} domain-specific
    knowledge and I managed to implement the second, globally, RegTAP search interface and make material
    contributions to the RegTAP specification itself to the effect that I was recognized as a co-author of the spec.\\
    \technologies{} Java, XPath 2.0 (Saxon), REST (Jersey), JDBC, Tomcat 7
  \item the \href{http://www.ivoa.net/documents/RegistryInterface/20091104/REC-RegistryInterface-1.0.pdf}{IVOA RI1} Search interface. This is a \href{https://en.wikipedia.org/wiki/SOAP}{SOAP interface}, defined in WSDL (i.e. following a top-down approach). 
    Due to the complexity of the XSDs used by the IVOA community I ran into problems with the automatic stub
    generation by tools like \href{https://docs.oracle.com/javase/6/docs/technotes/tools/share/xjc.html}{XJC} (which I have documented in \href{http://stackoverflow.com/q/17265960/274677}{SO} and raised
    against the tool's \href{https://java.net/jira/browse/JAXB-965}{JIRA}). So I decided to implement
    my own \href{https://docs.oracle.com/javase/6/docs/technotes/tools/share/wsimport.html}{wsimport}-like tool that reads WSDL and generates Java code (stubs and skeletons) to perform SOAP calls.\\
    \technologies{} Java, SOAP (Jersey), XSD, template-based code generation with \href{http://www.stringtemplate.org/.}{String Template}
  \item the \href{https://en.wikipedia.org/wiki/XML_Schema_(W3C)}{XML XSD schema-validation}
    and the IVOA validation of the various services.
    \\
    \technologies{} Java, XML, XSD, XPath 2.0 (Saxon), HTTP.
  \item the report, email notifications and visual plots generation subsystem.
    \\
    \technologies{} Java, \href{http://www.jfree.org/jfreechart/}{JFreeChart}
    \item the entire database schema for PostgreSQL. I wrote the DDL statements for the entire database which comprises 102 tables in 4 schemas and was responsible for the installation, fine-tuning, index creation, performance optimization, etc of the PostgreSQL database. Wrote the entire DB-facing \href{https://en.wikipedia.org/wiki/Database_abstraction_layer}{DAO} code in Java.\\
      \technologies{} PostgreSQL 9.2, PL/pgSQL, JDBC, SQL.
    \item Finally, I also handled all DevOps aspects including: installing and configuring Apache Tomcat7 and the PostgreSQL cluster, authoring the entire build system based on
      Ant and \href{http://ant.apache.org/ivy/}{Ivy}, writing test cases (JUnit), coding standards checking with \href{http://checkstyle.sourceforge.net/}{Checkstyle}, source code analyzers with
      \href{https://pmd.github.io/}{PMD} and \href{http://findbugs.sourceforge.net/}{FindBugs},
      code coverage with \href{http://cobertura.github.io/cobertura/}{Cobertura}, etc.
  \end{itemize}

}}

}

%------------------------------------------------

\headedsection
{\href{http://www.neuropublic.com}{Neuropublic S.A.}}
{\textsc{Piraeus, Greece}


\headedsubsection % Job title entry for the current employer
{Senior Software Engineer}
{June '12 -- March '13}
{\bodytext{
    I worked as a software engineer in the company's flagship \href{https://www.neuropublic.gr/en/component/k2/gaia-epicheirein}{GAIA} series of cloud-based services.
    \\
    Among the things I did:
    \begin{itemize}
    \item designed and developed all the server-side code for an expert system that advises farmers
      on the subsidies they are entitled to apply for. I devised a formal way to capture the legal requirements
      for each subsidy in the form of \href{https://en.wikipedia.org/wiki/S-expression}{Lisp S-expressions}.
      Essentially, each subsidy ministerial decision is
      modelled as a declarative, Lisp-like \href{https://en.wikipedia.org/wiki/Domain-specific_language}{DSL}.
      Then the system reads from the database the profile of each user (age, income, past subsidies, location
      and size of fields, crops, animal capital) and dynamically evaluates them against all possible subsidies.
      The system finally informs the user which subsidies (s)he can apply for and also creates
      a visual diagram that explains to the users why they failed to qualify for certain subsidies.
      S-expressions were used in order to allow for dynamic evaluation and also because the company wanted
      the DSL logic / script to be created by business people. So S-expression were chosen
      as Lisp has no other syntax except the opening and closing parenthesis.\\
      \technologies{} \href{https://clojure.org/}{Clojure} (for
      the dynamic evaluation of the S-expressions), dynamic evaluation of Clojure code from Java,
      \href{https://en.wikipedia.org/wiki/DOT_(graph_description_language)}{DOT} (for visualizing
      criteria logic using graphs), \href{https://en.wikipedia.org/wiki/IText}{iText} for creating
      PDF reports, \href{http://www.stringtemplate.org/}{StringTemplate}, JDBC.
    \item designed and developed the authentication / authorization database and implemented all back-end Java
      code to integrate Apache Shiro into our \href{http://www.javaserverfaces.org/}{JSF} pages and deliver an elaborate RBAC authentication solution
      for all GAIA applications.\\
      \technologies{} \href{http://shiro.apache.org/}{Apache Shiro}, PostgreSQL, JDBC.
    \item implemented a \href{http://www.jython.org/}{Jython} solution integrating Java code with
      Python scripts. The objective was to
      allow the accounting people to write simple arithmetic expressions (formulas) to derive dynamically
      various fields that were used in the generation of financial reports.
      \\
      \technologies{} \href{http://www.jython.org/}{Jython}, Python, \href{http://community.jaspersoft.com/project/jasperreports-library}{Jasper Reports}.
    \end{itemize}
}}
}

\headedsection
{\href{http://www.synelixis.com}{Synelixis Solutions}}
{\textsc{Athens, Greece}} {

  \headedsubsection % Job title entry for the current employer
{Senior Software Engineer at the \href{http://www.synelixis.com/coast/}{COAST} project}
{Feb '10 -- March '12}
{\bodytext{
    I led the company's technical participation in this project. The domain of the project was multimedia
    content delivery and streaming over a dynamically configured mesh of network caches that are co-located
    with network routers (something \textit{like} a content delivery network).
    Some of the things I did:
    \begin{itemize}
    \item designed and implemented: (a) the cache communication protocol on top of TCP and (b) custom HTTP proxies
      in Java that transparently take advantage of the caches. Also, for demonstration purposes, coded
      a few pages for video streaming solutions.
      \\ \technologies{} Java TCP networking, HTML, integration with video streaming / playback.
    \item designed and implemented a rich client overview and monitoring application that visualizes
      the utilization and available capacity of all network caches, plots graphs over time and allows
      an operator to dynamically interact with them by issuing commands or writing Python scripts
      (to automate common administrative logic) in a console-like prompt.
      \\ \technologies Java (networking), Java Swing, Jython (for the dynamic console)
    \end{itemize}
}}
  
\headedsubsection % Job title entry for the current employer
{Senior Software Engineer at the \href{http://www.innovationseeds.eu/Virtual_Library/Results/BEYWATCH.kl}{BeyWatch} project}
{Jan '09 -- Oct '10}
{\bodytext{
    I led the company's technical participation in this project. The domain of the project was home automation
    and centralized optimization and scheduling of household tasks involving use of electrical appliances
    in a system of continuously changing real-time tariffs.
    Some of the things I did:
    \begin{itemize}
    \item implemented a brute-force particle swarm optimization algorithm in Java that accepts as inputs
      a set of tasks, electricity tariffs over the next 24 hours and a weather forecast (so that solar
      panel output or heating needs may be anticipated) and calculates the best possible scheduling of
      tasks to minimize either cost or carbon footprint or a combination of both.
      \\ \technologies{} Java
    \item implemented a rich client Java application that visualizes the scheduling of the various
      tasks against the electricity tariffs
      \\ \technologies{} Java Swing
    \item implemented the client (scheduler) side of a number of controllers that are used to program or monitor
      household appliances (washing machines, refrigerators, solar panels)
      \\ \technologies{} Java, REST, \href{https://www.osgi.org/}{OSGI}.
    \end{itemize}
}}
}

%------------------------------------------------

\headedsection
{\href{http://www.semantix.gr}{Semantix}}
{\textsc{Athens, Greece}} {

\headedsubsection % Job title entry for the current employer
{Technical Project Manager for the EGOS Visualization Tool}
{March '08 -- Sep '08}
{\bodytext{
I was technical project manager for the EGOS Visualization Tool for ESOC (Contract C21283). The project implemented a graphical front-end that integrates and provides additional functionality on top of the existing tools used by ESOC to check compliance with coding standards and conventions.
}}

\headedsubsection % Job title entry for the current employer
{Software Engineer in \href{http://www.tapeditor.com/}{Roaming Studio} products}
{2005 -- 2007}
{\bodytext{
    I was responsible for the implementation of the critical 
    conversions functionality in both the Roaming Studio and the Roaming Components products. These products
    are the company's flagship telecom products and revolve around the processing of TAP
    files which are used in \href{https://en.wikipedia.org/wiki/Roaming}{GSM Roaming}.
    The conversion logic is responsible for converting, e.g., a TAP file of version TAP3.11 into a file
    of version TAP3.10. The files are defined using
    different \href{https://en.wikipedia.org/wiki/Abstract_Syntax_Notation_One}{ASN.1} grammars both from
    a syntax and a semantics point of view so
    the transformation is a complex business logic procedure. Since there are more than 7 different \href{http://what-when-how.com/roaming-in-wireless-networks/transferred-account-procedures-billing-and-settlement/}{TAP3}
    versions and conversion had to take place between any arbitrary pair there was
    a large number of conversion routines to be written. I defined a
    \href{https://en.wikipedia.org/wiki/Domain-specific_language}{DSL} used to describe the logic
    behind these conversions (essentially transformation algorithms on deeply nested ASN.1 trees) and built a Python
    script that automatically generated C++ code implementing these transformations. In total I was able to
    reduce tens of thousands of lines of C++ code into as little as 700 conversion rules expressed in the
    above DSL
    (plus the Python script code generator).
    \\
    \technologies{} ASN.1, C++, Python
}}

\headedsubsection % Job title entry for the current employer
{Software Engineer at project ATLAS}
{2002 -- 2004}
{\bodytext{
    I was part of the engineering team that was tasked to implement the new Vodafone Greece billing system.
    Data (customer, subscriber, bills and
    calls information) migration to the new billing system and integration with over a dozen legacy
    peripheral systems which were scheduled to survive the migration to the new billing system.
    I was responsible for the integration code which was implemented in
    C++ and Java (for higher-level daemons) and which undertook to sustain the information flow between the
    new billing system and legacy systems which were not going to be replaced. Since no modifications were
    permitted in the legacy systems, the new billing system had to be wrapped in a fa\c{c}ade allowing it to
    expose the same interface towards the legacy systems.
    I was also involved with the design of a massive ``mediating'' staging database of more
    than 400 tables and 40,000 lines of \href{https://en.wikipedia.org/wiki/PL/SQL}{PL-SQL} code
    (a large percentage of which I implemented myself).
    Finally I had small exposure with \href{https://en.wikipedia.org/wiki/Portal_Software}{Portal Infranet} programming
    (now \href{http://www.orafaq.com/wiki/BRM}{Oracle Billing and Revenue Management}).
    \\
    \technologies{} Java, Oracle, PL-SQL, C++, C
}}

\headedsubsection % Job title entry for the current employer
{Software Engineer at project DCH}
{2001 -- 2002}
{\bodytext{
    Design and implementation of the Vodafone Greece Data Clearing House system for roaming calls.
    I implemented code in C++ (for the decoding / encoding and validations / transformations
    of the \href{http://what-when-how.com/roaming-in-wireless-networks/transferred-account-procedures-billing-and-settlement/}{TAP} records), Java (for the orchestration logic) and PL/SQL (for server-side processing)
    \\
    \technologies{} C++, ASN.1, Java, Oracle, PL-SQL.
}}


}
%------------------------------------------------


%------------------------------------------------

\begin{center}
%\textit{Please refer to \href{http://www.linkedin.com/in/ciesbreijs}{my Linkedin profile} for the complete list of work experiences along with recommendations.}
\end{center}

%------------------------------------------------

\spacedhrule{-0.2em}{-0.4em} % Horizontal rule - the first bracket is whitespace before and the second is after

%----------------------------------------------------------------------------------------
%	EDUCATION SECTION
%----------------------------------------------------------------------------------------

\roottitle{Education} % Top level section

\headedsection % Employer name which can include a hyperlink and location/URL on the right side of the page
{National Technical University of Athens}
{\textsc{Athens, Greece}} {

\headedsubsection % Job title entry for the current employer
{Ph.D. from the Department of Electrical and Computer Engineering}
{1997 -- 2001}
{\bodytext{NTUA is the oldest and most prestigious technical university in Greece.
    I was awarded a Ph.D. for research into applying modern (at that time) software technologies
    in telecommunications networks. During that period of time I worked mostly with C++ and Java and developed
    code using distributed technologies like CORBA, RMI; Java applets, code interacting with telecom switch
    equipment, developed specialized video streaming applications in C++, etc.
    During that period of time I also published about
    a dozen papers (3 in international refereed journals including IEEE, the rest in conferences). Publications
    list provided at the end.
    }}
}

\headedsection % Employer name which can include a hyperlink and location/URL on the right side of the page
{Technical University of Patras}
{\textsc{Patras, Greece}} {

\headedsubsection % Job title entry for the current employer
{Five-year Engineering Diploma from the Computer Engineering and Informatics Department}
{1992 -- 1997}
{\bodytext{
    CEID is the oldest university department in Greece focusing exclusively in computer and software engineering
    and awarding an engineering diploma after 5 years of studies.
    I entered 1\ts{st} in rank after a nationwide competitive examination and finished 2\ts{nd} in rank
    (class size \textasciitilde{150}).
    Coded various applications in technologies like Pascal, Lisp, C and Java.
    }}
}


\spacedhrule{0.5em}{-0.4em} % Horizontal rule - the first bracket is whitespace before and the second is after

%----------------------------------------------------------------------------------------
%	SKILLS SECTION
%----------------------------------------------------------------------------------------

\roottitle{Skills} % Top level section

\inlineheadsection % Special section that has an inline header with a 'hanging' paragraph
{Technical specialties:}
{Software design and implementation, alone or in a team. I am a professional Java coder but also love
  and have written production code in C, C++ and (more recently) Python and Clojure.
  Also, non-production code in JavaScript and OCaml. I am very comfortable
  at the command line and casually write Bash one-liners daily.
 Solid knowledge of the following technologies: Java 7, JDBC, \acr{XML}, \acr{XPath}, \acr{XSD}, \acr{REST}, \acr{SOAP}, RDBMSs,
  bare-bones JavaScript.
  Linux administration skills:\ bash, Apache Tomcat,
  Postgres\acr{SQL}}

\inlineheadsection{}{}
%------------------------------------------------

\inlineheadsection % Special section that has an inline header with a 'hanging' paragraph
{Natural languages:}
{Greek \textit{(mother tongue)}, English \textit{(full professional proficiency: Cambridge Certificate of Proficiency in English, Grade A)}, Spanish \textit{(elementary)}}

\spacedhrule{1.6em}{-0.4em} % Horizontal rule - the first bracket is whitespace before and the second is after

%----------------------------------------------------------------------------------------
%	REFERENCES SECTION
%----------------------------------------------------------------------------------------

\roottitle{References} % Top level section
Available upon request.
\iffalse
  \inlineheadsection
      {\href{mailto:Christophe.Arviset@esa.int}{Christophe ARVISET}}
      {Head of the \href{http://www.cosmos.esa.int/web/esdc}{Science Archives Unit}, \href{http://www.esa.int/About\_Us/ESACESAC}{ESAC}}

      \inlineheadsection{}{}

  \inlineheadsection
      {\href{mailto:i_koufoudakis@c-gaia.gr}{Yannis KOUFOUDAKIS}}
      {Managing Director, \href{https://www.c-gaia.gr/}{GAIA}.
        \\ GAIA is a 50 MEuro turnover company running the informational systems handling
        practically all EU-agricultural subsidies in Greece.}
\fi
\spacedhrule{1.6em}{-0.4em} % Horizontal rule - the first bracket is whitespace before and the second is after

%----------------------------------------------------------------------------------------
%	REFERENCES SECTION
%----------------------------------------------------------------------------------------

\roottitle{Publications} % Top level section
\paragraph{Books and reference works}

  \begin{description}
  \item[2001] Iakovos S. Venieris, Menelaos K. Perdikeas
    \href{http://onlinelibrary.wiley.com/doi/10.1002/0471219282.eot257/full}{``Distributed Intelligent Network''} article in the
    \href{http://eu.wiley.com/WileyCDA/WileyTitle/productCd-0471369721.html}{``Encyclopedia of Telecommunications''} ISBN: 0-471-36972, J. Proakis ed. John Wiley. 2002. pp. 719-729.
  \item[2000] Co-author in \href{http://eu.wiley.com/WileyCDA/WileyTitle/productCd-0471623792.html}{``Object Oriented Software Technologies in Telecommunications: from theory to practice''}, edited by I.Venieris, F.Zizza, T. Magedanz. Published by John Wiley \& Sons LTD, Chichester, UK, April 2000.
  \item[1999] M. K. Perdikeas, O. I. Pyrovolakis, F. G. Chatzipapadopoulos and I. S. Venieris, ``Service Design in Distributed Intelligent Networks'' in ``On the Way to the Information Society --- A Retrospective View on 5 Years of ACTS IS\&N Research'' Baltzer press, 1999.
  \end{description}
  \paragraph{Papers in International Journals}
  \begin{description}
  \item[2001] M.K. Perdikeas and I.S. Venieris,  ``Parlay-based Service Engineering in a Converged Internet-PSTN Environment'' \href{http://www.sciencedirect.com/science/journal/13891286}{Computer Networks (Elsevier)},
    vol. 35, Issue 5, April 2001, pp. 565--578
  \item[2000]F. G. Chatzipapadopoulos, M. K. Perdikeas and I. S. Venieris, ``Mobile Agent and CORBA Technologies in the Broadband Intelligent Network'', \href{http://www.comsoc.org/commag}{IEEE Communications Magazine.}, Vol. 38, Issue 6, pp. 116--124
    \item[1999] M.K. Perdikeas, F.G. Chatzipapadopoulos, I.S. Venieris and G. Marino, ``Mobile Agent Standards and Available Platforms'', \href{http://www.sciencedirect.com/science/journal/13891286}{Computer Networks (Elsevier)}, vol. 31, Issue 19, August 1999, pp. 1999--2016
  \end{description}
\paragraph{Papers in International Conferences, Workshops etc. (non exclusive list)}
\begin{description}
\item[2014] Christophe Arviset, Menelaus Perdikeas et al., ``The Euro-VO Registry, re-engineering the back-end'' (\href{http://adsabs.harvard.edu/abs/2015ASPC..495..457A}{link to abstract}) in the \href{http://www.adass2014.org/announcements_en.php}{24\textsuperscript{th} annual conference on Astronomical Data Analysis Software and Systems}, Calgary---Canada
\item[2011] Theodore Zahariadis, Menelaos Perdikeas, Fotis Chatzipapadopoulos, Javier Lucio Ruiz Andino,
  Maria Angeles Barba Rodriguez: Middleware for energy aware appliances
  2nd Workshop on eeBuildings Data Models, Sofia Antipolis---France
\item[2010] Menelaos Perdikeas, Theodore Zahariadis, and Pierre Plaza: The BeyWatch Conceptual Model for Demand-Side Management, E-Energy conference, October 14--15, Athens---Greece
\item[2001] I. S. Venieris; T. Magedanz; M. Perdikeas; L. Hagen: Enhancing Parlay with mobile code technologies---IEEE 2001 Intelligent Network Workshop, 6-9 May 2001, Boston MA, pp. 287--299
\item[2001] M.K.Perdikeas et al: Realizing Distributed Intelligent Networks Based on Distributed Object and Mobile Agent Technologies---First International Conference on Networking, ICN 2001, LNCS 2094, Colmar, France July 9-13, 2001, Proceedings Part II, pp. 488--496
\item[1999] M.K.Perdikeas et al., An evaluation study of mobile agent technology: standardization, implementation and evolution---IEEE International Conference on Multimedia Computing and Systems, July 1999, Florence, pp. 287--291, vol.2
%\item[1999] Designing Advanced Services for Distributed Intelligent Broadband Networks
%  Pyrovolakis O., Chatzipapadopoulos F., Perdikeas M. and Venieris Iakovos S.
%  ---Proceedings of the 7\textsuperscript{th} International Conference on Advances in Communications and Control (COMCON7)
% \item \ldots and eight (8) others in previous years including a couple of IEEE and \href{http://www.icin.co.uk/}{ICIN}
  conferences.
  
\end{description}

\spacedhrule{1.6em}{-0.4em} % Horizontal rule - the first bracket is whitespace before and the second is after

%----------------------------------------------------------------------------------------
%	INTERESTS SECTION
%----------------------------------------------------------------------------------------

\roottitle{Interests} % Top level section

\inlineheadsection % Special section that has an inline header with a 'hanging' paragraph
{Non-exhaustive:}
{optimizing my environment, setup and workflow; Emacs, JavaScript, ReactJS, Linux, i3 window manager, \LaTeX; historical reading, Chess, boardgames.}

%----------------------------------------------------------------------------------------

\end{document}
