%%%%%%%%%%%%%%%%%%%%%%%%%%%%%%%%%%%%%%%%%
% Cies Resume/CV
% LaTeX Template
% Version 1.1 (20/7/14)
%
% This template has been downloaded from:
% http://www.LaTeXTemplates.com
%
% Original author:
% Cies Breijs (cies@kde.nl)
% https://github.com/cies/resume with extensive modifications by:
% Vel (vel@latextemplates.com)
%
% License:
% CC BY-NC-SA 3.0 (http://creativecommons.org/licenses/by-nc-sa/3.0/)
%
%%%%%%%%%%%%%%%%%%%%%%%%%%%%%%%%%%%%%%%%%

%----------------------------------------------------------------------------------------
%	PACKAGES AND OTHER DOCUMENT CONFIGURATIONS
%----------------------------------------------------------------------------------------

\documentclass[10pt,a4paper]{article} % Font size (10-12pt) and paper size (a4paper, letterpaper, legalpaper, etc)

% Copyright (c) 2012 Cies Breijs
%
% The MIT License
%
% Permission is hereby granted, free of charge, to any person obtaining a copy
% of this software and associated documentation files (the "Software"), to deal
% in the Software without restriction, including without limitation the rights
% to use, copy, modify, merge, publish, distribute, sublicense, and/or sell
% copies of the Software, and to permit persons to whom the Software is
% furnished to do so, subject to the following conditions:
%
% The above copyright notice and this permission notice shall be included in
% all copies or substantial portions of the Software.
%
% THE SOFTWARE IS PROVIDED "AS IS", WITHOUT WARRANTY OF ANY KIND, EXPRESS OR
% IMPLIED, INCLUDING BUT NOT LIMITED TO THE WARRANTIES OF MERCHANTABILITY,
% FITNESS FOR A PARTICULAR PURPOSE AND NONINFRINGEMENT. IN NO EVENT SHALL THE
% AUTHORS OR COPYRIGHT HOLDERS BE LIABLE FOR ANY CLAIM, DAMAGES OR OTHER
% LIABILITY, WHETHER IN AN ACTION OF CONTRACT, TORT OR OTHERWISE, ARISING FROM,
% OUT OF OR IN CONNECTION WITH THE SOFTWARE OR THE USE OR OTHER DEALINGS IN THE
% SOFTWARE.

%%% LOAD AND SETUP PACKAGES

\usepackage[margin=0.75in]{geometry} % Adjusts the margins

\usepackage{multicol} % Required for multiple columns of text

\usepackage{mdwlist} % Required to fine tune lists with a inline headings and indented content

\usepackage{relsize} % Required for the \textscale command for custom small caps text

\usepackage{hyperref} % Required for customizing links
\usepackage{xcolor} % Required for specifying custom colors
\definecolor{dark-blue}{rgb}{0.15,0.15,0.4} % Defines the dark blue color used for links
\hypersetup{colorlinks,linkcolor={dark-blue},citecolor={dark-blue},urlcolor={dark-blue}} % Assigns the dark blue color to all links in the template

\usepackage{tgpagella} % Use the TeX Gyre Pagella font throughout the document
\usepackage[T1]{fontenc}
\usepackage{microtype} % Slightly tweaks character and word spacings for better typography

\pagestyle{empty} % Stop page numbering

%----------------------------------------------------------------------------------------
%	DEFINE STRUCTURAL COMMANDS
%----------------------------------------------------------------------------------------

\newenvironment{indentsection} % Defines the indentsection environment which indents text in sections titles
{\begin{list}{}{\setlength{\leftmargin}{\newparindent}\setlength{\parsep}{0pt}\setlength{\parskip}{0pt}\setlength{\itemsep}{0pt}\setlength{\topsep}{0pt}}}{\end{list}}

\newcommand*\maintitle[2]{\noindent{\LARGE \textbf{#1}}\ \ \ \emph{#2}\vspace{0.3em}} % Main title (name) with date of birth or subtitle

\newcommand*\roottitle[1]{\subsection*{#1}\vspace{-0.3em}\nopagebreak[4]} % Top level sections in the template

\newcommand{\headedsection}[3]{\nopagebreak[4]\begin{indentsection}\item[]\textscale{1.1}{#1}\hfill#2#3\end{indentsection}\nopagebreak[4]} % Section title used for a new employer

\newcommand{\headedsubsection}[3]{\nopagebreak[4]\begin{indentsection}\item[]\textbf{#1}\hfill\emph{#2}#3\end{indentsection}\nopagebreak[4]} % Section title used for a new position

\newcommand{\bodytext}[1]{\nopagebreak[4]\begin{indentsection}\item[]#1\end{indentsection}\pagebreak[2]} % Body text (indented)

\newcommand{\inlineheadsection}[2]{\begin{basedescript}{\setlength{\leftmargin}{\doubleparindent}}\item[\hspace{\newparindent}\textbf{#1}]#2\end{basedescript}\vspace{-1.7em}} % Section title where body text starts immediately after the title

\newcommand*\acr[1]{\textscale{.85}{#1}} % Custom acronyms command

\newcommand*\bull{\ \ \raisebox{-0.365em}[-1em][-1em]{\textscale{4}{$\cdot$}} \ } % Custom bullet point for separating content

\newlength{\newparindent} % It seems not to work when simply using \parindent...
\addtolength{\newparindent}{\parindent}

\newlength{\doubleparindent} % A double \parindent...
\addtolength{\doubleparindent}{\parindent}

\newcommand{\breakvspace}[1]{\pagebreak[2]\vspace{#1}\pagebreak[2]} % A custom vspace command with custom before and after spacing lengths
\newcommand{\nobreakvspace}[1]{\nopagebreak[4]\vspace{#1}\nopagebreak[4]} % A custom vspace command with custom before and after spacing lengths that do not break the page

\newcommand{\spacedhrule}[2]{\breakvspace{#1}\hrule\nobreakvspace{#2}} % Defines a horizontal line with some vertical space before and after it % Include structure.tex which contains packages and document layout definitions

\hyphenation{Some-long-word} % Specify custom hyphenation points in words with dashes where you would like hyphenation to occur, or alternatively, don't put any dashes in a word to stop hyphenation altogether

\newcommand{\technologies}[0]{\textbf{\textit{Technologies:}}}
\begin{document} 

%----------------------------------------------------------------------------------------
%	NAME AND CONTACT INFORMATION
%----------------------------------------------------------------------------------------

\maintitle{Menelaus Perdikeas}{December 24, 1974}  % Your name and date of birth or subtitle

\noindent\href{mailto:mperdikeas@gmail.com}{mperdikeas@gmail.com}\bull % Your email address
%\textsmaller{+}30 (210) 5322914\bull
perdikeas \textit{(Skype)}\bull % Your phone number(s) and Skype username
\href{http://mperdikeas.github.com}{mperdikeas.github.com}\\ % Your URL
11, Dimitras st.\bull Ag. Paraskeuh, 15342\bull Athens\bull Greece % Your address

\spacedhrule{0.9em}{-0.4em} % Horizontal rule - the first bracket is whitespace before and the second is after

%----------------------------------------------------------------------------------------
%	SUMMARY SECTION
%----------------------------------------------------------------------------------------

\roottitle{Summary} % Root section title

\vspace{-1.3em} % Reduce whitespace after the Summary heading and the two-column content

\begin{multicols}{2}  % Start a two-column layout
  \noindent
  %\textit{A summary of your interests, achievements, history, topic of study or any other short summary of your professional life.}\\\\
  I have been coding in Java since version JDK 1.1 came out in 1997. Over the years I have written more than 200
  LoC in Java and have explored a wide array of technologies and frameworks in the landscape. 
  I am a proficient Emacs user (with an Emacs setup that I've honed over the years for maximum productivity)
  I like to live as close to the command line as possible
  and have no need of an IDE to hold my hand or ``help'' me in any way (though I can of course also use an IDE
  if required). I am a full stack Java engineer and have worked professionally in practically the entire
  stack from front-facing web services, rich client applications, server-side Java code all the way to the back-end relational
  database. I am also typically involved in devops stuff as well. That is, in addition to writing the code I also
  write the build system (Ant / Ivy), test cases, coding standards compliance checking, code coverage,
  integration testing, shel-scripting, etc.
  The browser is the only part of the stack where I haven't worked in a professionally capacity
  though I have ample experience in JavaScript in my playground projects and am currently exploring ReactJS in my
  free time.

\end{multicols}

\spacedhrule{0.5em}{-0.4em} % Horizontal rule - the first bracket is whitespace before and the second is after

%----------------------------------------------------------------------------------------
%	EXPERIENCE SECTION
%----------------------------------------------------------------------------------------

\roottitle{Experience} % Top level section

\headedsection % Employer name which can include a hyperlink and location/URL on the right side of the page
    {\href{http://www.esac.com}{ESAC} (through \href{http://www.neuropublic.gr}{Neuropublic})}
{\textsc{European Space Agency / ESAC, Madrid, Spain}} {

\headedsubsection % Job title entry for the current employer
{Software Architect and Lead Programmer}
{March '13 -- June '15}
{\bodytext{I was the software architect and the sole coder for the
    \href{registry2.euro-vo.org}{new EuroVO Registry system}
    which is scheduled to replace the \href{registry.euro-vo.org}{existing EuroVO Registry system} which has been
    serving the global \href{http://www.ivoa.net/}{IVOA} community for the last ten (10) years.

  I worked for two years with people from the \href{http://www.sciops.esa.int/index.php?project=SAT}{Science
    Archives Team at ESAC}, with no company supervision
  and designed and implemented the entire system from scratch with the exception of some parts
  of the front-end web-based GUI.
  Among others, I designed and implemented:
  \begin{itemize}
  \item the \href{http://www.openarchives.org/pmh/}{OAI-PMH} harvesting functionality. I couldn't find a reliable 3rd party OAI-PMH library
    so I implemented the entire protocol from scratch.\\
    \technologies{} Java, XPath 2.0 (Saxon), XSLT, REST (Jersey), HTTP
  \item the \href{http://www.ivoa.net/documents/RegTAP/}{RegTAP} functionality and interface. I came into the project with \textbf{zero} domain-specific
    knowledge and I managed to implement the second, globally, RegTAP search interface and make material
    contributions to the RegTAP specification itself to the effect that I was recognized as a co-author of the spec.\\
    \technologies{} Java, XPath 2.0 (Saxon), REST (Jersey), JDBC
  \item the \href{http://www.ivoa.net/documents/RegistryInterface/20091104/REC-RegistryInterface-1.0.pdf}{IVOA RI1} Search interface. This is SOAP interface, definined in WSDL (i.e. following a top-down approach). 
    Due to the complexity of the XSDs used by the IVOA community I ran into problems with the automatic stub
    generation by tools like wsimport (which I have documented and raised against). So I decided to implement
    my own wsimport-like tool that reads WSDL and generates Java code (stubs and skeletons) to perform SOAP calls.\\
    \technologies{} Java, SOAP (Jersey), XSD, template-based code generation with \href{http://www.stringtemplate.org/.}{String Template}
  \item the \href{https://en.wikipedia.org/wiki/XML_Schema_(W3C)}{XML XSD schema-validation}
    and the IVOA validation of the various services.
    \\
    \technologies{} Java, XML, XSD, XPath 2.0 (Saxon), HTTP.
  \item the report, email notifications and visual plots generation subsystem.
    \\
    \technologies{}: Java, JFreeChart
    \item the entire database schema for PostgreSQL. I wrote the DDL statements for the entire database which comprises 102 tables in 4 schemas and was responsible for the installation, fine-tuning, index creation, performance optimization, etc of the PostgreSQL database. I also wrote the entire DB-facing \href{https://en.wikipedia.org/wiki/Database_abstraction_layer}{DAO} code.\\
      \technologies{} PostgreSQL 9.2, JDBC.
    \item I also handled all DevOps aspects including: authoring the entire build system based on
      Ant and \href{http://ant.apache.org/ivy/}{Ivy}, writing test cases (JUnit), coding standards checking with \href{http://checkstyle.sourceforge.net/}{Checkstyle}, source code analyzers with
      \href{https://pmd.github.io/}{PMD} and \href{http://findbugs.sourceforge.net/}{FindBugs},
      code coverage with \href{http://cobertura.github.io/cobertura/}{Cobertura}, etc.
  \end{itemize}

}}

}

%------------------------------------------------

\headedsection
{\href{http://www.neuropublic.com}{Neuropublic S.A.}}
{\textsc{Piraeus, Grece}


\headedsubsection % Job title entry for the current employer
{Senior Software Engineer}
{June '12 -- March '13}
{\bodytext{
    I worked as a software engineer in the company's flagship \href{https://www.neuropublic.gr/en/component/k2/gaia-epicheirein}{GAIA} series of cloud-based services.
    \\
    Among the things I did:
    \begin{itemize}
    \item designed and developed all the server-side code for an expert system that advises farmers
      on the subsidies they are entitled to apply for. I devised a formal way to capture the legal requirements
      for each subsidy in the form of \href{https://en.wikipedia.org/wiki/S-expression}{Lisp S-expressions}.
      Essentially each subsidy ministerial decision is
      transformed into a Lisp-like \href{https://en.wikipedia.org/wiki/Domain-specific_language}{DSL}.
      Then the system reads from the database the profile of each user (age, income, past subsidies, location
      and size of fields, crops, animal capital) and dynamically evaluates them against all possible subsidies.
      The system finally informs the user which subsidies (s)he can apply for and also creates
      a visual diagram that explains to the users why the failed to qualify for certain subsidies.
      S-expressions were used in order to allow for dynamic evaluation and also because the company wanted
      the DSL for each ministerial decision to be created by business people. So S-expression were chosen
      as Lisp has no other syntax except the opening and closing parenthesis.\\
      \technologies{} \href{https://clojure.org/}{Clojure} (for
      the dynamic evaluation of the S-expressions), dynamic evaluation of Clojure code from Java,
      \href{https://en.wikipedia.org/wiki/DOT_(graph_description_language)}{DOT} (for visualizing
      subsidies criteria logic using graphs), \href{https://en.wikipedia.org/wiki/IText}{iText} for creating
      PDF reports, \href{http://www.stringtemplate.org/}{StringTemplate}, JDBC.
    \item designed and developed the authentication / authorization database and implemented all back-end Java
      code to integrate Apache Shiro into our \href{http://www.javaserverfaces.org/}{JSF} pages and deliver an elaborate RBAC authentication solution
      for all GAIA applications.\\
      \technologies{} \href{http://shiro.apache.org/}{Apache Shiro}, PostgreSQL, JDBC.
    \item implemented a \href{http://www.jython.org/}{Jython} solution integrating Java code with
      Python scripts. The objective was to
      allow the accounting people to write simple arithmetic expressions (formulas) to derive dynamically
      various fields that were used in the generation of financial reports.
      \\
      \technologies{} \href{http://www.jython.org/}{Jython}, Python, \href{http://community.jaspersoft.com/project/jasperreports-library}{Jasper Reports}.
    \end{itemize}
}}
}

\headedsection
{\href{http://www.synelixis.com}{Synelixis Solutions}}
{\textsc{Athens, Greece}} {

  \headedsubsection % Job title entry for the current employer
{Senior Software Engineer at the \href{http://www.synelixis.com/coast/}{COAST} project}
{Feb '10 -- March '12}
{\bodytext{
    I led the company's technical participation in this project. The domain of the project was multimedia
    content delivery and streaming over a dynamically configured mesh of network caches that are co-located
    with network routers (something \textit{like} a content delivery network).
    Some of the things I did:
    \begin{itemize}
    \item designed and implemented the cache communication protocol on top of TCP and custom HTTP proxies
      in Java that transparently take advantage of the caches. Also, for demonstration purposes coded
      a few pages for video streaming solutions.
      \\ \technologies{} Java TCP networking, HTML, integration with video streaming / playback.
    \item designed and implemented a rich client overview and monitoring application that visualizes
      the utilization and available capacity of all network caches, plots graphs over time and allows
      an operator to dynamically interact with them by issuing commands or writing Python scripts
      in a console-like prompt.
      \\ \technologies Java Swing, Jython (for the dynamic console)
    \end{itemize}
}}
  
\headedsubsection % Job title entry for the current employer
{Senior Software Engineer at the \href{http://www.innovationseeds.eu/Virtual_Library/Results/BEYWATCH.kl}{BeyWatch} project}
{Jan '09 -- Oct '10}
{\bodytext{
    I led the company's technical participation in this project. The domain of the project was home automation
    and centralized optimization and scheduling of household tasks involving use of electrical appliances
    in a system of continuously changing real-time tarriffs.
    Some of the things I did:
    \begin{itemize}
    \item implemented a brute-force particle swarm optimization algorithm in Java that accepts as inputs
      a set of tasks, electricity tarriffs over the next 24 hours and a weather forecast (so that solar
      panel output or heating needs may be anticipated) and caclulates the best possible scheduling of
      tasks to minimize either cost or carbon footprint or a combination of both.
      \\ \technologies{} Java
    \item implemented a rich client Java application that visualizes the scheduling of the various
      tasks against the electricity tarrifs
      \\ \technologies{} Java Swing
    \item implemented the client (scheduler) side of a number of controllers that are used to program or monitor
      household appliances (washing machines, refrigerators, solar panels)
      \\ \technologies{} Java, REST, \href{https://www.osgi.org/}{OSGI}.
    \end{itemize}
}}
}

%------------------------------------------------

\headedsection
{\href{http://www.semantix.gr}{Semantix}}
{\textsc{Athens, Greece}} {

\headedsubsection % Job title entry for the current employer
{Technical Project Manager for the EGOS Visualization Tool}
{March '08 -- Sep '08}
{\bodytext{
I was technical project manager for the EGOS Visualization Tool for ESOC (Contract C21283). The project implemented a graphical front-end that integrates and provides additional functionality on top of the existing tools used by ESOC to check compliance with coding standards and conventions.
}}

\headedsubsection % Job title entry for the current employer
{Software Engineer in \href{http://www.tapeditor.com/}{Roaming Studio} products}
{2005 -- 2007}
{\bodytext{
    I was responsible for the implementation of the critical 
    conversions functionality in both the Roaming Studio and the Roaming Components products. These products
    are the company’s flagship telecom products and revolve around the processing of TAP
    files which are used in \href{https://en.wikipedia.org/wiki/Roaming}{GSM Roaming}.
    The conversion logic is responsible for converting, e.g., a TAP file of version TAP3.11 into a file
    of version TAP3.10. The files are defined using
    different \href{https://en.wikipedia.org/wiki/Abstract_Syntax_Notation_One}{ASN.1} grammars both from
    a syntax and a semantical point of view so
    the transformation is a complex business logic procedure. Since there are more than 7 different TAP3
    versions and conversion had to take place between any arbitrary pair (both upwards and downwards) there was
    a large number of conversion routines to be written. I defined a meta-language used to describe the logic
    behind these conversions (essentially transformation algorithms on huge ASN.1 trees) and built a Python
    script that automatically generated C++ code implementing these transformations. In total I was able to
    reduce tens of thousands of lines of C++ code into as little as 700 “meta-language” conversion rules
    (plus the Python script code generator).
    \\
    \technologies{} ASN.1, C++, Python
}}

\headedsubsection % Job title entry for the current employer
{Software Engineer at project ATLAS}
{2002 -- 2004}
{\bodytext{
    Design and implementation of the new Vodafone Greece billing system. Data (customer, subscriber, bills and
    calls information) migration to the new billing system and integration with over a dozen legacy
    peripheral systems which will survive the migration to the new billing system.
    I was responsible for the integration code which was implemented in
    C++ and Java (for higher-level demons) and which undertook to sustain the information flow between the
    new billing system and the existing legacy systems which were not replaced. Since no modifications were
    permitted in the legacy systems, the new billing system had to be wrapped in a façade allowing it to
    project the interface of the old billing system towards the legacy systems. In this way the legacy systems
    would continue to operate unaltered and under the illusion that they are still interacting with the old
    billing system. I was also involved with the design of a massive ``mediating'' staging database of more
    than 400 tables and 40,000 lines of \href{https://en.wikipedia.org/wiki/PL/SQL|{PL-SQL} code
    (a large percentage of which I implemented myself).
    Finally I had small exposure with \href{https://en.wikipedia.org/wiki/Portal_Software}{Portal Infranet} programming.
    \\
    \technologies{} Java, Oracle, PL-SQL, C++
}}


}
%------------------------------------------------


%------------------------------------------------

\begin{center}
%\textit{Please refer to \href{http://www.linkedin.com/in/ciesbreijs}{my Linkedin profile} for the complete list of work experiences along with recommendations.}
\end{center}

%------------------------------------------------

\spacedhrule{-0.2em}{-0.4em} % Horizontal rule - the first bracket is whitespace before and the second is after

%----------------------------------------------------------------------------------------
%	EDUCATION SECTION
%----------------------------------------------------------------------------------------

\roottitle{Education} % Top level section

\headedsection % Employer name which can include a hyperlink and location/URL on the right side of the page
{Erasmus University Rotterdam}
{\textsc{Rotterdam, The Netherlands}} {

\headedsubsection % Job title entry for the current employer
{Bachelor degree in Computer Science \& Economics}
{2004 -- 2007}
{\bodytext{Focussed on the economics of open source, rapid application development (\acr{RAD}) and the semantic web technology stack (\acr{RDF}/\acr{RDFS}, \acr{OWL} and \acr{SPARQL}).}}
}

%------------------------------------------------

\headedsection % Employer name which can include a hyperlink and location/URL on the right side of the page
{Technical University Delft}
{\textsc{Delft, The Netherlands}} {

\headedsubsection % Job title entry for the current employer
{Industrial Design Engineering \textnormal{(discontinued)}}
{2001 -- 2002} {}
}

%------------------------------------------------

\headedsection % Employer name which can include a hyperlink and location/URL on the right side of the page
{Libanon Lyceum}
{\textsc{Rotterdam, The Netherlands}} {

\headedsubsection % Job title entry for the current employer
{\acr{VWO} \textnormal{(pre-university secondary education)}}
{1994 -- 2000} {}
}

\spacedhrule{0.5em}{-0.4em} % Horizontal rule - the first bracket is whitespace before and the second is after

%----------------------------------------------------------------------------------------
%	SKILLS SECTION
%----------------------------------------------------------------------------------------

\roottitle{Skills} % Top level section

\inlineheadsection % Special section that has an inline header with a 'hanging' paragraph
{Technical specialties:}
{Software design and implementation, with(in) a team. I love Ruby/Python/Java/C++ and flirt regularly with Haskell. Solid knowledge of web technologies:\ \acr{HTML+CSS}, \acr{XML}, \acr{RDF}, \acr{REST}, \acr{SOAP} and JavaScript (mainly jQuery). Linux administration skills:\ bash, Apache, My\acr{SQL}, Postgres\acr{SQL}, virtualization/cloud (Open\acr{VZ}, \acr{VM}ware, \acr{KVM}, Xen and \acr{EC}2), datacenter automation (Puppet and Chef), continuous integration (Hudson/Jenkins).}

%------------------------------------------------

\inlineheadsection % Special section that has an inline header with a 'hanging' paragraph
{Natural languages:}
{Dutch \textit{(mother tongue)}, English \textit{(full professional proficiency)}, German \textit{(limited working proficiency)}, French \textit{(elementary proficiency)} and Mandarin Chinese \textit{(beginner)}.}

\spacedhrule{1.6em}{-0.4em} % Horizontal rule - the first bracket is whitespace before and the second is after

%----------------------------------------------------------------------------------------
%	REFERENCES SECTION
%----------------------------------------------------------------------------------------

\roottitle{References} % Top level section

  \inlineheadsection
      {Christophe Arviset}
      {asdfasdfasdf}
  \inlineheadsection
      {Markus Demleitner}
      {asdfasdfasdf}

\spacedhrule{1.6em}{-0.4em} % Horizontal rule - the first bracket is whitespace before and the second is after
%----------------------------------------------------------------------------------------
%	INTERESTS SECTION
%----------------------------------------------------------------------------------------

\roottitle{Interests} % Top level section

\inlineheadsection % Special section that has an inline header with a 'hanging' paragraph
{Non-exhaustive:}
{obsessively optimizing my environment, setup and workflow, Emacs, JavaScript, ReactJS, Linux, org-mode, i3 window manager, \LaTeX; historical reading}

%----------------------------------------------------------------------------------------

\end{document}
