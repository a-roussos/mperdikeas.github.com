\documentclass[11pt, twocolumn]{scrartcl}
%\documentclass[11pt, twocolumn]{article}
\newenvironment{italicquotes}
{\begin{quote}\itshape}
{\end{quote}}
\usepackage[greek,english]{babel}
\usepackage[utf8x]{inputenc}
\usepackage[hidelinks]{hyperref}
%\usepackage[none]{hyphenat}

\setlength{\parskip}{\baselineskip}%
\setlength{\columnsep}{0.75cm} 
\begin{document}
\begin{otherlanguage}{greek}
\title{Κουλτούρα ελευθερίας ή μιζέριας?}
\subtitle{Το μεγαλύτερο πρόβλημα της Ελλάδας είναι η αντικαπιταλιστική της κουλτούρα}


\begin{hyphenrules}{greek}
\hyphenation{\detokenize{^^65^^d0-^^6e^^61^^69}}
\end{hyphenrules}

\author{\latintext Russell Lamberti\footnote{Μετάφραση άρθρου που δημοσιεύτηκε στις 8 Ιουλίου 2015 στην ιστοσελίδα
\latintext \url{www.mises.org}.}}
\date{}
\maketitle

%\begin{abstract}
%The abstract text goes here.
%\end{abstract}
Θεωρείται πολιτικά απρεπές να επικρίνεται η κουλτούρα ενός λαού στους καιρούς μας, αλλά ανεξάρτητα από το εάν χρησιμοποιούνται ευρώ ή δραχμές, εντός ή εκτός της Ευρωπαϊκής Ένωσης, η Ελλάδα πάσχει κατά κάποιο τρόπο από ένα είδος πολιτισμικής δυσλειτουργίας. 
Δεν μιλώ για τα ήθη, τις παραδόσεις, την αρχιτεκτονική ή την μουσική της, και σίγουρα δεν μιλάμε για το φαγητό της. Μιλώ για τον πολιτισμικά αποκρυσταλλωμένο αντικαπιταλισμό της.

Οι διαπραγματεύσεις, συμφωνίες, αντιπροσφορές, δημοψηφίσματα, διαδηλώσεις και τα συμπαρομαρτούντα δεν θα οδηγήσουν πουθενά εάν οι Έλληνες δεν αποβάλλουν από μέσα τους τον κρατισμό και εάν δεν ανακαλύψουν ξανά την Ελληνική αριστεία στο επιχειρείν.

Αυτό φαίνεται καθαρά στην περίπτωση της Αργεντινής.

Θα υπέθετε κανείς ότι μια χρεοκοπία και μια κρίση δημόσιου χρέους δρα πάντα παιδαγωγικά και επανακατευθύνει ένα έθνος σε μια λογική προσανατολισμένη στην ελεύθερη αγορά καθώς οι άφρονες προσδοκίες ευημερίας με πυλώνα ένα μεγάλο κράτος και ένα μοντέλο σοσιαλισμού χρηματοδοτούμενου με δανεικά διαψεύδονται πανηγυρικά.
Ωστόσο η Αργεντινή, δεκατρία χρόνια μετά την χρεοκοπία του 2002, και μετά από χρόνια ραγδαίας αύξησης του πληθωρισμού, ελλείψεις συναλλάγματος και οικονομικής δυσπραγίας, εξακολουθεί να προσκολλάται στους παντελώς ανίδεους, υπερ-κρατιστές, σοσιαλιστές δυνάστες της, οι οποίοι συνεχίζουν να κατεδαφίζουν την οικονομία. 
Ο λόγος είναι ότι η κυρίαρχη κουλτούρα της, ο πυρήνας της πολιτισμικής σκέψης, δεν άλλαξε ποτέ.

Όταν η κουλτούρα είναι τοξική, το πάνω γίνεται κάτω, το μαύρο γίνεται άσπρο και η σοσιαλιστική αποτυχία ερμηνεύεται ως καπιταλιστική αποτυχία.

Στο έργο του \latintext''The Anti-Capitalistic Mentality'' \greektext ο \latintext\ Ludwig von Mises \greektext περιέγραψε αυτόν τον πολιτισμικό αντι-καπιταλισμό ως εξής:
\begin{italicquotes}
Ο Γιάνης νομίζει ότι όλες αυτές οι νέες βιομηχανίες που τον προμηθεύουν με προϊόντα τεχνολογίας και άνεσης, άγνωστα στον πατέρα του, προήλθαν από κάποια μυθική δύναμη που ονομάζεται «πρόοδος».

Η συσσώρευση κεφαλαίου, η επιχειρηματικότητα και η τεχνολογική εφευρετικότητα δεν συνεισφέρουν τίποτα σε αυτήν την αυθόρμητη δημιουργία της ευημερίας.

Σύμφωνα με αυτά που ο Γιάνης πιστεύει σχετικά με την αύξηση της παραγωγικότητας της εργασίας, αν κάποιος πρέπει να πιστωθεί για το αποτέλεσμα αυτό, τότε είναι μόνον ο άνθρωπος που εργάζεται στη γραμμή παραγωγής.
\end{italicquotes}

Οι γεννήτορες της παραπάνω αντίληψης για την καπιταλιστική βιομηχανία εξυμνούνται από τα πανεπιστήμια σαν οι μεγαλύτεροι φιλόσοφοι και ευεργέτες της ανθρωπότητας και οι διδασκαλίες τους γίνονται δεκτές με ευλαβικό δέος από εκατομμύρια ανθρώπων που τα σπίτια τους, εκτός από άλλες συσκευές, είναι εξοπλισμένα με τα πιο σύγχρονα οπτικοακουστικά μέσα.
Ο μεγαλύτερος κίνδυνος για την Ελλάδα δεν είναι η λιτότητα, ή χρεοκοπία, το ευρώ ή η δραχμή. Και σίγουρα δεν είναι ο μπαμπούλας της έξωσης από τις πιστωτικές αγορές κρατικών τίτλων - είναι ότι η ελληνική νοοτροπία παραμένει εχθρική στην ελεύθερη και αδέσμευτη αγορά και είναι χρονίως ερωτευμένη με τον κρατισμό.

Πάρτε μια άλλη χώρα της Λατινικής Αμερικής, την  Βενεζουέλα.

Αφού βίωσε καταστροφικό πληθωρισμό τις δεκαετίες του 1980 και '90, στις εκλογές του 1998 το εκλογικό σώμα ανέδειξε στην εξουσία έναν άλλο κρατιστή, υπέρμαχο του πληθωρισμού, τον Ούγκο Τσάβες. Επανεξελέγη το 2000, το 2006 και το 2012, και κατόπιν εξελέγη ο παρόμοιος διάδοχός του Νικολάς Μαδούρο το 2013 ακόμη και όταν η χώρα βρισκόταν σε κατάσταση υπερπληθωριστικού σπιράλ θανάτου ολοταχώς προς οριστική οικονομική κατάρρευση.

Το πρόβλημα της Βενεζουέλας δεν είναι τελικά η δημοσιονομική κακοδιαχείριση - είναι η αντι-καπιταλιστική της νοοτροπία.
Έτσι είναι και με την Ελλάδα.

Ύστερα από ήδη εξασφαλισμένη ελάφρυνσης του χρέους και αναδιάρθρωση του υπολοίπου για τα επόμενα πενήντα χρόνια με επιδοτούμενο επιτόκιο (κάτι που ισοδυναμεί με επίσημη αποδοχή χρεοκοπίας) και μετά από την επίτευξη κάποιας ισχνής οικονομικής ανάπτυξης το 2014 μέσω μείωσης των φόρων και ελαφριάς συρρίκνωσης του  αρτηριοσκληρωτικού και υπερτροφικού δημόσιου τομέα, η τοξική Ελληνική κουλτούρα επικράτησε για μια ακόμη φορά και ανέδειξε στην εξουσία μια ομάδα σκληροπυρηνικών σοσιαλιστών για να σύρουν την χώρα πίσω στο τέλμα.
Το γεγονός ότι από την άλλη πλευρά του τραπεζιού βρίσκεται επίσης μια ομάδα οπαδών του κεντρικού σχεδιασμού (στην ΕΕ, το ΔΝΤ και την ΕΚΤ) επιτείνει την σύγχυση και δυσχεραίνει την αναγνώριση της Ελληνικής παθογένειας.

Η Ελλάδα βρίσκεται έρμαιο σε μια διελκυστίνδα ανάμεσα σε δύο πόλους που αμφότεροι εν τέλει εμφορούνται από κρατικιστικές αντιλήψεις, επειδή οι πολίτες της προκρίνουν την διεκδίκηση ή την υπεράσπιση αργομισθιών, ρυθμιστικών προνομίων και προσόδων αντί της ελευθερίας.

\textbf{Οι περισσότερες χώρες κατά καιρούς βιώνουν υφεσιακές κρίσεις αλλά ορισμένες ανακάμπτουν καλύτερα από άλλες.}

Οποιαδήποτε χώρα ενδέχεται να ξοδέψει υπερβολικά και να αντιμετωπίσει δημοσιονομικά προβλήματα, και οι περισσότερες το έχουν κάνει.
Δεν έχει περάσει πολύς καιρός που η Βρετανία αναγκάστηκε να καταφύγει στην επαιτεία προς το ΔΝΤ το 1976 και να εκχωρήσει τη δημοσιονομική της διαχείριση στο εν λόγω ίδρυμα. Κατά το δεύτερο ήμισυ της δεκαετίας του '70 η Βρετανία ήταν εντελώς χάλια.

Η Αμερική χρεοκόπησε άτυπα το 1971 αθετώντας τις διεθνείς υποχρεώσεις της και υπέστη μια κυλιόμενη πληθωριστική οικονομική κρίση για το υπόλοιπο της δεκαετίας του 1970. 
Και οι δύο αυτές χώρες εν τέλει ανέκαμψαν.

Έτσι έγινε και με την Χιλή, την Ουρουγουάη, και τις Φιλιππίνες μετά την δημοσιονομική και χρηματοπιστωτική κρίση τους τις δεκαετίας του '70 και του '80.
Αλλά μερικές δεν ανέκαμψαν, και πιστεύω ότι αυτό συμβαίνει όταν η εθνική κουλτούρα είναι, ή έχει γίνει, ουσιαστικά ριζικά αντι-καπιταλιστική και αποδέχεται αξιοθρήνητα την αργή σήψη στην σκιά ενός υπερτροφικού κράτους που ελέγχει κάθε πτυχή της οικονομικής ζωής.

Εκτός από την Αργεντινή και τη Βενεζουέλα, έχουμε δει παρατεταμένη οικονομική και χρηματοπιστωτική αστάθεια μετά από επώδυνες κρίσεις σε χώρες όπως η Ζιμπάμπουε, Γκάνα, Βολιβία, Νιγηρία, Ρωσία, Τουρκία και τώρα χώρες στην νότια Ευρώπη.
Αυτές οι χώρες δεν φαίνεται να μαθαίνουν από τα λάθη τους, επειδή δεν θέλουν να αναζητήσουν ή αδυνατούν να εντοπίσουν το μάθημα καθώς η κουλτούρα τους βρίσκεται βαλτωμένη μέσα στη θολούρα της συγκεχυμένης σκέψης τους.

Αλλά πραγματικά το μάθημα είναι σαφές.

Μια οικονομική κρίση μπορεί να τραντάξει μια χώρα που είναι κατά βάση φιλο-καπιταλιστική (ή επί το πλείστον φιλο-καπιταλιστική) που έχει προσωρινά χάσει το δρόμο της και να την επαναφέρει πίσω στο σωστό δρόμο.
Αλλά δεν υπάρχει καμία προοπτική ανάκαμψης, όταν η κουλτούρα της χώρας έχει διαβρωθεί πλήρως από έναν παιδαριώδη αντι-καπιταλισμό, προσκόλληση στον κρατισμό, και αντανακλαστική εχθρότητα προς τον επιχειρηματικό δυναμισμό και την ατομική πρωτοβουλία.

Γι 'αυτές τις χώρες, η κρίση δεν μπορεί να μεταστραφεί σε ανάκαμψη, αλλά αντίθετα γίνεται προάγγελος μιας χρόνιας και ακόμα βαθύτερης εθνικής παρακμής. 
Μόνο μια αλλαγή κουλτούρας που θα προκύψει από διάδοση υγιών αρχών και αντιλήψεων μπορεί να κάνει την Ελλάδα (και άλλες χώρες) ένα γόνιμο έδαφος για την αποδοχή των πραγματικών λύσεων.

Η ανάγκη να διαδοθεί το καλό μήνυμα της ελευθερίας και της ελεύθερης αγοράς είναι σαφώς πιο επείγουσα από ποτέ.


\end{otherlanguage}
\end{document}
